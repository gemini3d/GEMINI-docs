\documentclass[11pt,letterpaper]{article}

\usepackage[pdftex]{graphicx}
\usepackage{natbib}
\usepackage{fullpage}
\usepackage{lineno}
\usepackage{multirow}
\usepackage{wrapfig}
\usepackage{amsmath}
\usepackage{amssymb}
\usepackage{sidecap}
\usepackage{hyperref}

\begin{document}

\setlength{\parindent}{0mm}
\setlength{\parskip}{0.4cm}

\bibliographystyle{apalike}

%\modulolinenumbers[5]
%\linenumbers

\title{\textbf{GEMINI} test descriptions}
\author{Matthew D. Zettergren, PhD\\ Associate Professor of Engineering Physics\\ Center for Space and Atmospheric Physics\\ Physical Sciences Department \\Embry-Riddle Aeronautical University\\mattzett@gmail.com\\zettergm@erau.edu}
\maketitle

\tableofcontents

\pagebreak


\section{Purpose of this document}

The \textbf{G}eospace \textbf{E}nvironment \textbf{M}odel of \textbf{I}on-\textbf{N}eutral \textbf{I}nteractions (GEMINI) is a general-purpose, three-dimensional (3D) terrestrial ionospheric model capable of describing most processes relevant to the ionosphere at medium to small spatial scales (200 m to 10000 km).  The main source code repository for GEMINI can be found at \url{https://github.com/gemini3d/GEMINI}.  This document describes the formulation of tests used to verify the GEMINI build and functioning.  


\section{Diffusion solver test problem}

As discussed in the formulation document \url{https://github.com/gemini3d/GEMINI-docs/blob/master/formulation/GEMINI.pdf}, the parabolic portions of the energy equations are solved using implicit finite difference methods (including TRBDF2 and backward Euler).  These are tested via solution of a simple heat equation describing the evolution of temperature $T(z,t)$ in space and time subject to uniform thermal conduction:
\begin{equation}
\frac{\partial T}{\partial t} - \lambda \frac{\partial^2 T}{\partial z^2} = 0.
\end{equation}
For purposes of testing we solve this equation on the \emph{bounded} domain $0 \le x \le 1$.  Invoking separation of variables we presume $T(z,t)=Z(z) \mathcal{T}(t)$ and substitute back into the original equation:  
\begin{equation}
\frac{1}{\lambda \mathcal{T}} \frac{\partial \mathcal{T}}{\partial t} - \frac{1}{Z} \frac{\partial^2 Z}{\partial z^2} = 0.
\end{equation}
Each term depends solely on one of the independent variables $x,t$, which implies that for this relation to be valid for all $x,t$ then each term must be equal to a constant.  
\begin{eqnarray}
\frac{1}{\lambda \mathcal{T}} \frac{d \mathcal{T}}{d t} &=& -k^2 \\
- \frac{1}{Z} \frac{d^2 Z}{d z^2} &=& k^2.
\end{eqnarray}
Note also that since we have dependence on only one variable that we have converted the derivatives into \emph{ordinary} derivatives.  The solutions to these ODEs read:
\begin{eqnarray}
\mathcal{T}(t) &=& A e^{-k^2 \lambda t} \\
Z(z) &=& A' \sin kz + B' \cos kz
\end{eqnarray}
The elemental solution ($k$ arbitrary) is given by: 
\begin{equation}
\tilde{T}(z,t)=Z(z) \mathcal{T}(t) = e^{-k^2 \lambda t} \left( A'' \sin kz + B'' \cos kz \right)
\end{equation}

Further progress toward a general solution requires specific initial and boundary conditions.  For our test problem we assume that the temperature goes to zero on the boundaries ($z \in \{0,1\}$).  Let us also assume that the initial temperature of the system is given by:  $T(z,0) = f(z)$.  First employing the condition $T(0,t)=0$, we find that $B''=0$.  The other boundary condition $T(1,t)=0$ sets restrictions on the argument/eigenvalues of the sine function, namely that $k=n \pi, n \in \mathbb{Z}^+$ is a set of roots for the sine function.  The elemental solution is then:
\begin{equation}
\tilde{T}_n(z,t)= A_n e^{-n^2 \pi^2 \lambda t} \sin \left( n \pi z \right)
\end{equation}
Any integer value chosen for $n$ results in a legitimate solution for the original partial differential equation; therefore, the general solution is a linear superposition of all such solutions.  
\begin{equation}
T(z,t)= \sum_{n=1}^{\infty} A_n ~ e^{-n^2 \pi^2 \lambda t} \sin \left( n \pi z \right).
\end{equation}
The initial condition can now be applied to solve for the coefficients $A_n$ by leveraging orthogonality of the sine functions.  That is by making use of the fact that:
\begin{equation}
\left< \sin(n \pi z) | \sin (n' \pi z) \right> = \int_0^1 \sin(n \pi z) \sin (n' \pi z) dz = \frac{1}{2} \delta_{nn'},
\end{equation}
we may produce a solution for $A_n$ from the series solution.  The initial condition is represented in summation form (a Fourer sine series) as:  
\begin{equation}
T(z,0) = f(z) = \sum_{n=1}^{\infty} A_n ~ \sin \left( n \pi z \right).
\end{equation}
Taking the scalar product of both sides with $\sin (n' \pi z)$ gives:
\begin{equation}
\left< f(z) | \sin (n' \pi z) \right> = \sum_{n=1}^{\infty} A_n ~ \left< \sin(n \pi z) | \sin (n' \pi z) \right> = \sum_n = \frac{A_n}{2} \delta_{nn'} = \frac{A_{n'}}{2}
\end{equation}
Thus the coefficients are:  
\begin{equation}
A_{n'} = 2 \left< f(z) | \sin (n' \pi z) \right> = 2 \int_0^1 f(z) \sin (n' \pi z) dz
\end{equation}

For purposes of testing it is easiest to pick a test problem with boundary conditions that are represented by a finite sum - one way this can be accomplished is by choosing a boundary condition that is an eigenfunction for this particular problem, say:
\begin{equation}
 f(z) = \sin(2 \pi z).
\end{equation}
From this, and orthogonality of the sine function it follows that:
\begin{equation}
A_{n'} = \left\{ \begin{array}{cc} 0 & n \ne 2 \\ 1 & n=2  \end{array} \right. ,
\end{equation}
and that the general solution for this specific set of boundary and initial conditions is:  
\begin{equation}
T(z,t) = e^{-4 \pi^2 \lambda t} \sin \left( 2 \pi z \right)
\end{equation}

Energy diffusion in GEMINI is only performed in the field-aligned direction.  Thus, for a 3D system we are effectively just executing a sequence of 1D diffusion equations.  


\section{Potential solver test problem}

The elliptic potential solver is tested using a simplified 2D test problem, Laplace's equation:
\begin{equation}
\frac{\partial^2 \Phi}{\partial x^2} + \frac{\partial^2 \Phi}{\partial y^2} = 0, 
\end{equation}
on the domain $0 \le x \le 1, 0 \le y \le 1$ with the boundary conditions $\Phi(x,0)-=\Phi(0,y)=\Phi(1,y)=0, \Phi(x,1)=f(x)$.  Exploying separation of variables $\Phi(x,y)=X(x)Y(y)$ we find the ODE solutions for $X(x)$ and $Y(y)$ to be:
\begin{eqnarray}
X(x) &=& A \sin kx + B \cos kx \\
Y(y) &=& A' \sinh ky + B' \cosh ky
\end{eqnarray}
The boundary conditions dictate the following constraints:  $\Phi(0,y)=0 \implies B=0, \Phi(x,0) \implies B'=0, \Phi(1,y)=0 \implies k=n \pi$.  Thus we have the general solution:
\begin{equation}
\Phi(x,y) = \sum_n A_n \sinh (n \pi y) \sin (n \pi x)
\end{equation}
Again choosing our boundary conditions for this test problem so that only one term in the series survives we may choose:
\begin{equation}
f(x) = \sin(2 \pi x).
\end{equation}
Which gives:
\begin{equation}
f(x) = \sum_n A_n \sinh (n \pi) \sin (n \pi x),
\end{equation}
for the potential evaluated at the non-grounded boundary.  By orthogonality the coefficients are:  
\begin{equation}
A_n = \frac{2}{\sinh (n \pi)} \left< f(x) | \sin (n \pi x) \right>; 
\end{equation}
however the only nonzero coefficient occurs for $n=2$:
\begin{equation}
A_2 = \frac{2}{\sinh (2 \pi)} \left< \sin(2 \pi x) | \sin (2 \pi x) \right> = \frac{1}{\sinh(2 \pi)}; 
\end{equation}
The solution for this set of boundary conditions is, thus: 
\begin{equation}
\Phi(x,y) = \frac{\sinh (2 \pi y)}{\sinh(2 \pi)} \sin (2 \pi x)
\end{equation}

All potential solutions currently in GEMINI are two-dimensional so this particular test problem is representative of each.  


\section{Advection solver test problem}

The advection solver in GEMINI deals with problems of the form:
\begin{equation}
\frac{\partial \rho}{\partial t} + \frac{\partial}{\partial z} \left( \rho v \right) = 0,  
\end{equation}
and higher dimensional equivalents (implemented through directional splitting).  For constant velocity (assumed to be given) a simpler equation, which can be solved analytically, results.  
\begin{equation}
\frac{\partial \rho}{\partial t} + v \frac{\partial \rho}{\partial z} = 0,  
\end{equation}
This equation can be seen to be equivalent to the wave equation by differentiating with respect to time and space respectively giving:  
\begin{eqnarray}
\frac{\partial^2 \rho}{\partial t^2} + v \frac{\partial^2 \rho}{\partial t \partial z} &=& 0 \\
\frac{\partial^2 \rho}{\partial z \partial t} + v \frac{\partial^2 \rho}{\partial z^2} &=& 0
\end{eqnarray}
Eliminating the cross partial derivatives from these equations gives the familiar wave equation.  
\begin{equation}
\frac{\partial^2 \rho}{\partial t^2} - v^2 \frac{\partial^2 \rho}{\partial z^2} = 0,  
\end{equation}
The solution to this particular equation is a wave of the form:
\begin{equation}
\rho(z,t)=f(z-vt),
\end{equation}
where the function $f$ is arbitrary, generally speaking, but dictated by the specific initial conditions of the problem of interest.  This solution can be verified by direct substitution, or derived by separation of variables - analogous to the test problems above.  For an initial condition given by:
\begin{equation}
\rho(z,0)=e^{-\frac{z^2}{2 \sigma_z^2}}
\end{equation}
The solution at later times is:
\begin{equation}
\rho(z,t)=e^{-\frac{(z-vt)^2}{2 \sigma_z^2}}
\end{equation}
For testing purposes it is useful to numerical solve this on a periodic domain, e.g. $0 \le x \le 1$.

GEMINI advects mass, momentum, and energy in all three dimensions, representative of the equation:  
\begin{eqnarray}
\frac{\partial \rho}{\partial t} + v_x \frac{\partial \rho}{\partial x} + v_y \frac{\partial \rho}{\partial y} + v_z \frac{\partial \rho}{\partial z} &=& 0 %\\  
%\frac{\partial \rho}{\partial t} + \mathbf{v} \cdot \nabla \rho &=& 0 \\  
\end{eqnarray}
For initial conditions given by:
\begin{equation}
\rho(z,0)=e^{-\frac{x^2}{2 \sigma_x^2}}e^{-\frac{y^2}{2 \sigma_y^2}}e^{-\frac{z^2}{2 \sigma_z^2}}
\end{equation}
the solution at later time is:
\begin{equation}
\rho(z,t)=e^{-\frac{(x-v_x t)^2}{2 \sigma_x^2}}e^{-\frac{(y - v_y t)^2}{2 \sigma_y^2}}e^{-\frac{(z-v_z t)^2}{2 \sigma_z^2}}
\end{equation}


\section{Error reporting}

Please create an issue on our GitHub website \url{https://github.com/gemini3d/} if you find an error in our documentation or codes.  


\section{Contributors}

Major contributors to GEMINI source code and testing include:  M. Hirsch, G. Grubbs, and M. Burleigh.


\pagebreak
%\setcounter{page}{1}

%\bibliography{GEMINI.bib}


\end{document}
