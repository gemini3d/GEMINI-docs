\documentclass[11pt,letterpaper]{article}

\usepackage[pdftex]{graphicx}
\usepackage{natbib}
\usepackage{fullpage}
\usepackage{lineno}
\usepackage{multirow}
\usepackage{wrapfig}
\usepackage{amsmath}
\usepackage{amssymb}
\usepackage{sidecap}
\usepackage{hyperref}

\begin{document}

\setlength{\parindent}{0mm}
\setlength{\parskip}{0.4cm}

\bibliographystyle{apalike}

%\modulolinenumbers[5]
%\linenumbers

\title{\textbf{GEMINI} test descriptions}
\author{Matthew D. Zettergren, PhD\\ Associate Professor of Engineering Physics\\ Center for Space and Atmospheric Physics\\ Physical Sciences Department \\Embry-Riddle Aeronautical University\\mattzett@gmail.com\\zettergm@erau.edu}
\maketitle

\tableofcontents

\pagebreak


\section{Purpose of this document}

The \textbf{G}eospace \textbf{E}nvironment \textbf{M}odel of \textbf{I}on-\textbf{N}eutral \textbf{I}nteractions (GEMINI) is a general-purpose, three-dimensional (3D) terrestrial ionospheric model capable of describing most processes relevant to the ionosphere at medium to small spatial scales (200 m to 10000 km).  This document describes the formulation of tests used to verify the GEMINI build and functioning.  


\section{Error reporting}

Please create an issue on our GitHub website \url{https://github.com/gemini3d/}


\section{Contributors}

Major contributors to GEMINI source code and testing include:  M. Hirsch, G. Grubbs, and M. Burleigh.


\pagebreak
%\setcounter{page}{1}

%\bibliography{GEMINI.bib}


\end{document}
