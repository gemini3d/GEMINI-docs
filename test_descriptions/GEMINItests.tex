\documentclass[11pt,letterpaper]{article}

\usepackage[pdftex]{graphicx}
\usepackage{natbib}
\usepackage{fullpage}
\usepackage{lineno}
\usepackage{multirow}
\usepackage{wrapfig}
\usepackage{amsmath}
\usepackage{amssymb}
\usepackage{sidecap}
\usepackage{hyperref}

\begin{document}

\setlength{\parindent}{0mm}
\setlength{\parskip}{0.4cm}

\bibliographystyle{apalike}

%\modulolinenumbers[5]
%\linenumbers

\title{\textbf{GEMINI} test descriptions}
\author{Matthew D. Zettergren, PhD\\ Associate Professor of Engineering Physics\\ Center for Space and Atmospheric Physics\\ Physical Sciences Department \\Embry-Riddle Aeronautical University\\mattzett@gmail.com\\zettergm@erau.edu}
\maketitle

\tableofcontents

\pagebreak


\section{Purpose of this document}

The \textbf{G}eospace \textbf{E}nvironment \textbf{M}odel of \textbf{I}on-\textbf{N}eutral \textbf{I}nteractions (GEMINI) is a general-purpose, three-dimensional (3D) terrestrial ionospheric model capable of describing most processes relevant to the ionosphere at medium to small spatial scales (200 m to 10000 km).  The main source code repository for GEMINI can be found at \url{https://github.com/gemini3d/GEMINI}.  This document describes the formulation of tests used to verify the GEMINI build and functioning.  


\section{Diffusion solver test}

As discussed in the formulation document \url{https://github.com/gemini3d/GEMINI-docs/blob/master/formulation/GEMINI.pdf}, the parabolic portions of the energy equations are solved using implicit finite difference methods (including TRBDF2 and backward Euler).  These are tested via solution of a simple heat equation describing the evolution of temperature $T(x,t)$:
\begin{equation}
\frac{\partial T}{\partial t} - \lambda \frac{\partial^2 T}{\partial x^2} = 0.
\end{equation}
For purposes of testing we solve this equation on the \emph{bounded} domain $0 \le x \le 1$.  Invoking separation of variables we presume $T(x,t)=X(x) \mathcal{T}(t)$ and substitute back into the original equation:  
\begin{equation}
\frac{1}{\lambda \mathcal{T}} \frac{\partial \mathcal{T}}{\partial t} - \frac{1}{X} \frac{\partial^2 X}{\partial x^2} = 0.
\end{equation}
Each term depends solely on one of the independent variables $x,t$, which implies that for this relation to be valid for all $x,t$ then each term must be equal to a constant.  
\begin{eqnarray}
-\frac{1}{\lambda \mathcal{T}} \frac{d \mathcal{T}}{d t} &=& k^2 \\
\frac{1}{X} \frac{d^2 X}{d x^2} &=& -k^2.
\end{eqnarray}
Note also that since we have dependence on only one variable that we may convert the derivatives into \emph{ordinary} derivatives.


\section{Error reporting}

Please create an issue on our GitHub website \url{https://github.com/gemini3d/}


\section{Contributors}

Major contributors to GEMINI source code and testing include:  M. Hirsch, G. Grubbs, and M. Burleigh.


\pagebreak
%\setcounter{page}{1}

%\bibliography{GEMINI.bib}


\end{document}
